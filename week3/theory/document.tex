\documentclass[12pt]{article}
\usepackage{amsmath,amssymb}
\usepackage{amsmath}
\usepackage{mathtools}
\newcommand{\hattheta}{\hat{\theta}}
\newcommand{\hatmintexpect}{\hattheta - \mathbb{E}[\hattheta]}
\newcommand{\expectmintheta}{\mathbb{E}[\hattheta]- \hattheta}
\newcommand{\hatmintexpectsqr}{(\hattheta - \mathbb{E}[\hattheta])^2}
\DeclareMathOperator{\Hessian}{Hess}
\begin{document}
\section{Fundamentals of Unconstrained Optimization}
\subsection{}
\label{1a}
The gradient of the Rosenbrock function in $x_1$ and $x_2$ it the following
\begin{equation}
\frac{\partial f(x)}{\partial x_1} = 2(200x_1^{3} - 200x_1x_2+x_1-1) 
\end{equation}
\begin{equation}
\frac{\partial f(x)}{\partial x_2} = 200(x_2 - x_1^2)
\end{equation}
Every element of Hessian matrix is defines as follow
\begin{equation}
 H_{ij} \equiv \frac{\partial^{2} f}{\partial x_{i} \partial x_{j} }
\end{equation}
Therefore we need calculate as well the second derivatives and we get
\begin{align}
H(x_1,x_2) = \begin{bmatrix}
2(600x_1^2-200x_2+1) & -400x_1 \\
-400x_1 & 200
\end{bmatrix}
\end{align}
\subsection{}
\label{1b}
The gradient at $x^* = (1,1)$ $(\frac{\partial f(1,1)}{\partial x_1}, \frac{\partial f(1,1)}{\partial x_2})  = (0,0)$, while the Hessian matrix is
\begin{align}
H(x_1,x_2) = \begin{bmatrix}
802 & -400 \\
-400 & 200
\end{bmatrix}
\end{align}
The matrix is positive definite when $x^TAx >0$ for each nonzero real vector x and this imply that  $H_{1,1} > 0$ and  $\det(A) > 0$. Therefore
\begin{equation}
Det(H) = 802*200-400^2 = 400 > 0
\end{equation}
\subsection{}
Similarly as in \ref{1a} and \ref{1b} we calculate gradients and Hessian.
\begin{equation}
\frac{\partial f(x)}{\partial x_1} = 2x_1 + 8
\end{equation}
\begin{equation}
\frac{\partial f(x)}{\partial x_2} = -4x_2+12
\end{equation}
The gradient at $x^* = (-4,3)$ $(\frac{\partial f(-4,3)}{\partial x_1}, \frac{\partial f(-4,3)}{\partial x_2})  = (0,0)$, and (-4,3) is the only solution for gradient equal to zero. while the Hessian matrix is


\begin{align}
H(x_1,x_2) = \begin{bmatrix}
2 & 0 \\
0 & -4
\end{bmatrix}
\end{align}
Since the Hessian is indefinite ($det(H) <0$) it's a saddle point.

\section{Case study}
\subsection{}
Such area of the error surface is a plateau. Applying small changes to the parameters at such point would not change much the value of the error function, since the surface is flat.
Between the $100^{th}$ and $10000^{th}$ iteration, it seems that the network is stuck in a plateau. Escaping plateaus is a common problem in optimization, especially with networks with a lot of layers.
\subsection{}
We would like to Hessian Matrix to be positive definite so that we could determine if its a local minimum


\end{document}